\documentclass{article}
\usepackage{amsmath}
\usepackage{graphicx}
\usepackage{xcolor}
\usepackage{geometry}
\usepackage{fancyhdr}
\geometry{margin=1in}
\pagestyle{fancy}
\fancyhf{}

\begin{document}

\section*{Digital Design Through Arduino}
\subsection*{Questions}

\section*{7447}

37. In the circuit shown in Fig. 3.20, if $C = 0$, the expression for $Y$ is

\begin{enumerate}
\item $Y = A\overline{B} + \overline{AB}$
\item $Y = A + B$
\item $Y = \overline{A} + \overline{B}$
\item $Y = AB$
\end{enumerate}

\begin{center}
(GATE EC 2014)
\end{center}

\begin{center}
\includegraphics[width=1\textwidth]{37.png}\\
Fig. 3.20
\end{center}

44. In the circuit shown below in Fig. 3.26, X and Y are digital inputs, and Z is a digital output. The equivalent circuit is a \\
(GATE EE 2019)

\begin{enumerate}
\item NAND gate
\item NOR gate
\item XOR gate
\item XNOR gate
\end{enumerate}

\begin{center}
\includegraphics[width=1\textwidth]{44.png}\\
Fig. 3.26
\end{center}

\newpage

\section*{K-Map}

48. Find the Boolean logic realised by the following circuit in Fig. 4.23.

\begin{center}
(GATE EC 2010)
\end{center}

\begin{center}
\includegraphics[width=1\textwidth]{48.png}\\
Fig. 4.23
\end{center}

51. Find the logic function implemented by the circuit given below in Fig. 4.26.

\begin{center}
(GATE EC 2017)
\end{center}

\begin{center}
\includegraphics[width=0.5\textwidth]{51.png}\\
Fig. 4.26
\end{center}

\newpage

\section*{7474}


3. In the circuit in Fig. 5.3, the clock (Clk) frequency provided to the circuit is $500\mathrm{MHz}$. Starting from the initial value of the flip-flop outputs $Q2Q1Q0 = 111$ with $D2 = 1$, fine the time after which $Q2Q1Q0 = 100$. \\
(GATE EC 2021)

\begin{center}
\includegraphics[width=1\textwidth]{3.png}\\
Fig. 5.3
\end{center}

9. Consider a sequential digital circuit consisting of T flip-flops and D flip-flops as shown in Fig. 5.9. CLKIN is the clock input to the circuit. At the beginning, Q1, Q2 and Q3 have values 0, 1 and 1, respectively. Which of the given values of $(Q_{1}, Q_{2}, Q_{3})$ can NEVER be obtained with this digital circuit? \\
(GATE CS 2023)

\begin{enumerate}
\item $(0,0,1)$
\item $(1,0,0)$
\item $(1,0,1)$
\item $(1,1,1)$
\end{enumerate}

\begin{center}
\includegraphics[width=1\textwidth]{9.png}\\
Fig. 5.9
\end{center}

\end{document}
